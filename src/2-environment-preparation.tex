\chapter{项目环境准备}

\section{基准代码}
本系列的基准代码规则如下:
\begin{enumerate}
    \item \textbf{Linux}: 任何体系结构下高于5.10.20版本,低于6.0版本的Linux内核都是可行的选项,你可以从\url{https://www.kernel.org/}中选择合适的版本。对于\texttt{initramfs},你可以选择任何一个发行版。
    \item \textbf{QEMU}: 版本>=7。
    \item \textbf{EDK2}: 跟随\url{https://github.com/tianocore/edk2}选择合适的版本。其中Ubuntu\_GCC5并不意味着GCC版本必须是5。
\end{enumerate}
\section{QEMU安装(二选一)}
\subsection{软件包安装}
\begin{lstlisting}[language=bash]
sudo apt-get install qemu-system-x86
qemu-system-x86_64 --version #检查QEMU版本
\end{lstlisting}
\subsection{源码编译安装}
\begin{lstlisting}[language=bash]
# 安装依赖
sudo apt-get install git libglib2.0-dev libfdt-dev libpixman-1-dev zlib1g-dev ninja-build

# 下载编译
mkdir ~/kvm && cd ~/kvm
wget https://download.qemu.org/qemu-7.2.0.tar.xz
tar -xf qemu-7.2.0.tar.xz
cd qemu-7.2.0
./configure --target-list=x86_64-softmmu
make -j$(nproc)
\end{lstlisting}

\subsection{常见依赖问题}
\begin{itemize}
    \item \textbf{ERROR: Cannot find Ninja} → 执行 \texttt{sudo apt install ninja-build}
    \item \textbf{ERROR: glib-2.56 required} → 执行 \texttt{sudo apt install libglib2.0-dev}
    \item \textbf{Dependency "pixman-1" not found} → 执行 \texttt{sudo apt install libpixman-1-dev}
\end{itemize}

\section{Linux内核编译}
\begin{lstlisting}[language=bash]
# 安装依赖
sudo apt-get install git fakeroot build-essential ncurses-dev xz-utils libssl-dev bc flex libelf-dev bison
mkdir ~/kernel && cd ~/kernel
wget https://git.kernel.org/pub/scm/linux/kernel/git/stable/linux.git/snapshot/linux-5.x.x.tar.gz
tar -xf linux-5.x.x.tar.gz
cd linux-5.x.x

make defconfig    # 使用默认配置
make -j$(nproc)   # 开始编译
\end{lstlisting}

\textbf{编译输出文件}: \texttt{arch/x86/boot/bzImage}

在编译过程中,可能会发生编译错误的情况。这可能是 GCC 版本过高导致的。这里给出一种可行的组合: Linux-5.19.17, 使用 GCC-12 进行编译,操作如下:

\begin{lstlisting}[language=bash]
sudo apt install gcc-12 g++-12 # 安装gcc-12
sudo update-alternatives --install /usr/bin/gcc gcc /usr/bin/gcc-12 70
sudo update-alternatives --install /usr/bin/g++ g++ /usr/bin/g++-12 70 # 切换 gcc 默认版本
\end{lstlisting}
经过这些操作后,应该可以正常编译 Linux 内核了。

\section{BusyBox编译与initramfs制作}
\subsection{BusyBox编译安装}
这里可以采用其他发行版(例如Ubuntu发行版等)。
\begin{lstlisting}[language=bash]
mkdir ~/kvm && cd ~/kvm
wget https://busybox.net/downloads/busybox-1.x.x.tar.bz2
tar -xf busybox-1.x.x.tar.bz2
cd busybox-1.x.x

# 配置命令
make menuconfig
# 选中: Settings > Build Options > [*] Build static binary
# 取消: Shells > [ ] Job control

make -j$(nproc) && make install
\end{lstlisting}

\subsection{创建启动脚本}
\begin{lstlisting}[language=bash]
cd busybox-1.x.x/_install/
mkdir -p proc sys dev tmp
echo -e '#!/bin/sh\nmount -t proc none /proc\nmount -t sysfs none /sys\nmount -t tmpfs none /tmp\nmount -t devtmpfs none /dev\necho "Hello Linux!"\nexec /bin/sh' > init
chmod +x init
\end{lstlisting}

\subsection{打包文件系统}
\begin{lstlisting}[language=bash]
find . -print0 | cpio --null -ov --format=newc | gzip -9 > ../initramfs.cpio.gz
\end{lstlisting}

\section{EDK2编译与OVMF文件构建}
EDK2的可编译源码在git submodule里,因此相对简单可靠的方式就是直接按照官网所说的那样:
\begin{lstlisting}[language=bash]
git clone https://github.com/tianocore/edk2.git
cd edk2
git submodule update --init
\end{lstlisting}
在编译之前,需要在系统里安装一些基础的包(例如base-devel uuid、nasm、acpica),然后就可以编译BaseTools并运行setup脚本。
\begin{lstlisting}[language=bash]
make -C BaseTools -j$(nproc)
source edksetup.sh
\end{lstlisting}
还需编辑配置文件\texttt{Conf/target.txt}(参考\url{https://github.com/tianocore/tianocore.github.io/wiki/How-to-build-OVMF}):
\begin{lstlisting}
ACTIVE_PLATFORM       = OvmfPkg/OvmfPkgX64.dsc
TARGET_ARCH           = X64
TOOL_CHAIN_TAG        = GCC5
\end{lstlisting}
最后,执行:
\begin{lstlisting}[language=bash]
build #需要确保运行过edksetup.sh
\end{lstlisting}
就可以在\texttt{Build/OvmfPkgX64/DEBUG\_GCC5/FV/}目录找到\texttt{OVMF\_CODE.fd}和\texttt{OVMF\_VARS.fd}两个编译出的OVMF文件,以及合并的\texttt{OVMF.fd}。(OvmfPkgX64\_GCC5和build选项有关,对应\texttt{Conf/target.txt}里面的配置。)

\section{启动QEMU}
\begin{lstlisting}[language=bash]
qemu-system-x86_64 \
    -m 8G \ #需分配足够内存,否则无法进入系统!
    -kernel linux-5.x.x/arch/x86/boot/bzImage \ #linux内核路径
    -initrd busybox-1.x.x/initramfs.cpio.gz \ #initramfs文件路径
    -bios edk2/Build/OvmfX64/DEBUG_GCC5/FV/OVMF.fd \ #OVMF文件路径(暂不启用UEFI可以去掉,使用默认SeaBIOS)
    -append "init=/init console=ttyS0" \ #传递给内核的启动参数:运行启动脚本,配置控制台输出
    -nographic \ #使用命令行界面
    -enable-kvm #启用kvm加速
\end{lstlisting}

\section{验证}
\begin{itemize}
    \item 检查内核版本: \texttt{uname -a}
    \item 查看启动参数: \texttt{cat /proc/cmdline}
    \item 检查UEFI是否启用:\texttt{test -d /sys/firmware/efi \&\& echo efi || echo bios}
    \item 测试基本命令: \texttt{ls}, \texttt{mount}
\end{itemize}
